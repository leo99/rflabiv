\documentclass[a4paper, 12pt]{article}
\usepackage[english]{babel}
\usepackage[utf8]{inputenc}
\usepackage[T1]{fontenc}
\usepackage{lmodern}
\usepackage{hyperref}
\usepackage[numbers, sort&compress]{natbib}
\usepackage{calc}
\usepackage{nowidow}
\usepackage{color}
\usepackage{subcaption}
\usepackage{epsfig}
\usepackage{epstopdf}
\usepackage{verbatim}
\usepackage{enumitem}
\usepackage{graphicx}
\usepackage{amsfonts,amssymb,amsbsy} 
\usepackage{amsmath}
\usepackage{mathtools}
\usepackage{float}


\usepackage{pgfplots}
\usepackage{pgfplotstable}
\usepgfplotslibrary{units}
\usepgfplotslibrary{smithchart}
\pgfplotsset{compat=1.3,
	every axis/.append style={thick},
	tick label style={font=\small},
	label style={font=\small},
	legend style={font=\footnotesize}}


\newlength{\oneLine}
\setlength{\oneLine}{12pt}

\newlength{\eqMargin}
\newlength{\eqHorizMargin}
\newlength{\eqVertMargin}

\setlength{\eqMargin}{20mm}
\setlength{\eqHorizMargin}{\eqMargin}
\setlength{\eqVertMargin}{\eqMargin}

% Paper
\setlength{\paperwidth}{210mm}
\setlength{\paperheight}{297mm}

% Rid the extra space
\setlength{\hoffset}{-1in}
\setlength{\voffset}{-1in}
\addtolength{\hoffset}{\eqHorizMargin}
\addtolength{\voffset}{\eqVertMargin}

% Set margin from the page border (horizontal)
\setlength{\oddsidemargin}{0pt}
\setlength{\evensidemargin}{0pt}

% Header
\setlength{\topmargin}{0pt}
\setlength{\headheight}{0pt}
\setlength{\headsep}{18pt}
%\renewcommand{\headrulewidth}{0pt}

% Footer
\addtolength{\footskip}{18pt}
%\renewcommand{\footrulewidth}{0pt}

% Margin notes
\setlength{\marginparsep}{0pt}
\setlength{\marginparwidth}{0pt}

% Text
\setlength{\textwidth}{\paperwidth - \hoffset - \hoffset - 25.4mm - 25.4mm}
\setlength{\textheight}{\paperheight - \voffset - \topmargin - \headheight - \headsep - \footskip - \voffset - 25.4mm - 25.4mm}

%\setlength{\labelwidth}{20mm}

% Hyperref settings
\hypersetup{
    unicode=true,					% non-Latin characters in Acrobat's bookmarks
    pdftoolbar=true,				% show Acrobat's toolbar?
    pdfmenubar=true,				% show Acrobat's menu?
    pdffitwindow=false,				% window fit to page when opened
    pdfstartview={FitH},			% fits the width of the page to the window
    pdftitle={S-26.3120 Radio Engineering, laboratory course},	% title
    pdfauthor={Tuomas Leinonen} {Gaurav Khairkar} {Lasse Toivanen},	% author
    pdfsubject={Radio Engineering},	% subject of the document
    pdfcreator={LaTeX},				% creator of the document
    pdfproducer={Aalto},			% producer of the document
    pdfkeywords={radio} {amplifier} {measurement},	% list of keywords
    pdfnewwindow=true,				% links in new window
    colorlinks=true,				% false: boxed links; true: colored links
    linkcolor=black,				% color of internal links
    citecolor=black,				% color of links to bibliography
    filecolor=black,				% color of file links
    urlcolor=black					% color of external links
}

% Bad hyphenation
%\hyphenation{}

\newcommand{\dB}{\ensuremath{\mathrm{dB}}}

\definecolor{dkred}{rgb}{0.6, 0, 0}
\definecolor{dkgrn}{rgb}{0, 0.6, 0}
\definecolor{dkblue}{rgb}{0, 0, 0.6}

\newcommand*{\mysection}[1]{%
\newpage%
\section*{#1}%
\addcontentsline{toc}{section}{#1}%
}

\pagestyle{plain}


\begin{document}

\begin{titlepage}
\pagestyle{empty}
\begin{center}

\vspace*{30mm}
\noindent\LARGE{\textbf{S-26.3120 Radio Engineering, laboratory course}}

\vspace*{20mm}

\Large{\textbf{Lab 4: Amplifier}}\\

\vspace*{15mm}

\large{\textbf{Final Report}}\\
\vspace{15mm}
\large{\today}
	
\vspace*{30mm}
\large{
	\begin{tabular}{l l}
		\textbf{Group 1:} 				& \\
		Tuomas Leinonen 				& 84695P \\
		Gaurav Khairkar					& 398194 \\
		Lasse Toivanen					& 82209T
	\end{tabular}
}

\end{center}

\end{titlepage}

\mysection{Abstract}

Text

\newpage
\addcontentsline{toc}{section}{Contents}
\tableofcontents

\mysection{Symbols and abbreviations}

\subsection*{Symbols}

	\begin{description}[font=\rmfamily\mdseries, leftmargin=25.5mm, style=sameline, align=right, labelsep=5mm, itemsep=-2pt]
		\item[$R$]						resistance [$\Omega$]
		\item[$C$]						capacitance [F]
		\item[$L$]						inductance [H]
		\item[$f$] 						frequnecy [Hz]
		\item[$Z$] 						impedance [$\Omega$]
		\item[$Y$] 						admittance [S]
		\item[$z$] 						normalized impedance
		\item[$y$] 						normalized admittance
		\item[$\lambda$] 				wavelength [m]
		\item[$F$] 						noise figure [dB]
		\item[$G_0$] 					power gain, equal to $|S_{21}|$ [dB]
		\item[$G_\mathrm{A}$] 			available power gain [dB]
		\item[$G_\mathrm{T}$] 			transducer power gain [dB]
		\item[$G_\mathrm{P}$] 			operating power gain [dB]
		\item[$L$] 						attenuation [dB]
		\item[$S_{ij}$] 				scattering parameter from port $i$ to port $j$
		\item[$\mathit{BW}$] 			bandwidth [Hz]
		\item[$\mathit{IIP}_n$] 		$n$th order input-referred intermodulation intercept point [dBm]
		\item[$\mathit{IM}_n$] 			$n$th order intermodulation product [dBm]
		\item[$\mathit{ICP}$] 			1~dB input compression point [dBm]
		\item[$P$] 						signal power [dBm]
		\item[$N$] 						noise power [dBm]
		\item[$T$] 						temperature [K]
		\item[$w$] 						transmission line width [m]
		\item[$h$] 						substrate height [m]
		\item[$\tan \delta$]			``tangent delta'', a merit of substrate quality (attenuation)
		\item[$\epsilon_\mathrm{r}$]	relative permittivity of the substrate
		\item[$V$]						voltage [V]
		\item[$I$]						current [I]
		\item[$U$]						unilateral figure of merit [U]
		\item[$\Delta$]					determinant of a scattering matrix
		\item[$K$]						Rollet's stability factor
		\item[$\mu$]					stability factor
		\item[$\rho$]					reflection coefficient
		\item[$\textit{RL}$] 			return loss [dB]
		\item[$l$]						transmission line length
	\end{description}

\newpage
\subsection*{Abbreviations}

	\begin{description}[font=\rmfamily\mdseries, leftmargin=25.5mm, style=sameline, align=right, labelsep=5mm, itemsep=-2pt]
		\item[LNA]					low-noise amplifier
		\item[SMA]					SubMiniature version A, a type of coaxial connector
		\item[pHEMT]				pseudomophic high electron mobility transistor, a type of field-effect transistor
		\item[RF]					radio frequency
		\item[DC]					direct current
		\item[CS]					common source, a type of field-effect transistor configuration
		\item[IDCS]					inductively degenerated common source
		\item[PTH]					plated-through hole, a via
		\item[PCB]					printed circuit board
		\item[CAD]					computer-aided design
		\item[ADS]					Agilent Advanced Design System, a CAD software
	\end{description}

\newpage
\section{Introduction}

Topic/Field -- overview (one paragraph)

Topic/Field -- a closer look (one paragraph)

In the fourth lab exercise during the \textit{S-26.3120 Radio Engineering, laboratory course} 
a low-noise amplifier (LNA) is to be designed, fabricated and measured. The lab was divided into 
checkpoints as follows. First there was a individually completed pre-study, based on which 
the actual amplifier was to be designed in groups using computer-aided design (CAD) tools. 
For this purpose, Agilent Advanced Design System (ADS) was used.

Using ADS, we created a suitable microstrip layout for our amplifier. Next, a third-party 
manufacturer fabricated a printed circuit board (PCB) according to our specifications. 
The amplifier was then fabricated using this board, given transistor and miscellaneous 
components. The amplifier was then tested and measured rigorously. All of this was to be 
documented in a detailed lab report.

This document is the final report of this lab exercise and is organized ad follows. 
In the section following the introduction, theoretical background governing LNA design is 
presented briefly. In the third section, the preliminary design used as a the starting 
point of the group work is shown. Section \ref{s:design} gives explains the final amplifier 
designed with ADS in detail. In the two sections following this explanation, measurements 
and obtained results are presented. The second to last section draws the conclusion. 
Finally, in the last section, feedback and suggestions to improve this course are given.

\newpage
\section{Theoretical background}

Text

\newpage
\section{Preliminary design}

In this section, the preliminary design used as a starting point for the group work is presented. 
This presentation is intentionally rather short, presenting only the main points of the manual 
design procedure and the crude pre-design. For a more complete depiction of the derivation of 
the amplifier circuit shown in this section, the reader is advised to get acquanted with the 
pre-study reports.

The preliminary design is based solely on linear (and also otherwise ``ideal'') scattering 
parameter analysis; a process with a direct connection to the theory shown in the previous 
chapter and in textbooks like \cite{bahl, gonz, pozar}. The process is manual in the sense 
that no purpose-built CAD tools were used. Thus it is indeed possible to obtain same results 
with a ``back-of-the-envelope'' approach albeit extremely cumbersome. To help with this, 
a completely self-made Matlab-script was used.

A cicruit schematic of the pre-design is shown in Fig.~\ref{f:pre}. In the three tables following 
the figure, the key characteristics of the preliminary design are shown. Table~\ref{t:pre1} presents 
the reflection coefficients used for matching. These reflection coffecients are realized single 
shunt-stub matching with transmission line lengths given in Table~\ref{t:pre2}. In Table~\ref{t:pre3}, 
the obtained key performance metrics are compared with design goals. All design goals are met simultaneously.

\begin{figure}[!h]
	\centering
	\vspace*{\oneLine}
	\includegraphics[scale=1]{img/pre.png}
	\setlength{\unitlength}{1mm}
	\begin{picture}(1,1)
		\put(-71, 13.3){\small IN}
		\put(-56.5, 16.7){\small$0.177\;\lambda$}
		\put(-72.5, 6){\small$0.136\;\lambda$}
		\put(-34, 12){\small$100\;\Omega$}
		\put(-23.5, 19.5){\small$0.306\;\lambda$}
		\put(-7, 9){\small$0.135\;\lambda$}
		\put(-1, 16){\small OUT}
	\end{picture}
	\caption{Amplifier pre-design ($Z_0 = 50\;\Omega$).}
	\label{f:pre}
\end{figure}


\begin{table}[!h]
	\begin{center}
		\caption{Reflection coefficients of the preliminary design.}
		\label{t:pre1}
		\renewcommand{\arraystretch}{1.2}
		\begin{tabular}{lcccccc}
			\multicolumn{3}{l}{Reflection coefficient}	& & {Cartesian}	& & Polar 	\\
			\hline
			Source 	& & $\rho_\mathrm{S}$		& &	$-0.191 + \mathrm{j} 0.461$ & &  $0.499 \,\angle {+112.6 {\,}^\circ}$  	\\
			Input 	& & $\rho_\mathrm{in}$		& &	$-0.307 - \mathrm{j} 0.503$	& & $0.589 \,\angle {-121.5 {\,}^\circ}$  	\\
			Load 	& & $\rho_\mathrm{L}$		& &	$+0.461 + \mathrm{j} 0.171$	& & $0.492 \,\angle {+20.4 {\,}^\circ}$  	\\				
			Output 	& & $\rho_\mathrm{out}$		& &	$+0.461 - \mathrm{j} 0.171$	& & $0.492 \,\angle {-20.4 {\,}^\circ}$  				
		\end{tabular}
	\end{center}
	\vspace{-1\oneLine}
\end{table}

\begin{table}[!h]
	\begin{center}
		\caption{Matching circuits of the pre-design ($Z_0 = 50\;\Omega$).}
		\label{t:pre2}
		\renewcommand{\arraystretch}{1.2}
		\begin{tabular}{lccc}
			Port		&	Distance from transistor $l_1$ [$\lambda$]	& Stub length $l_2$ [$\lambda$]	\\
			\hline
			Input		&	0.177										& 0.136				\\
			Output		&	0.306										& 0.135			
		\end{tabular}
	\end{center}
	\vspace{-\oneLine}
\end{table}

\begin{table}[!h]
	\begin{center}
		\caption{Properties of the pre-design.}
		\label{t:pre3}
		\renewcommand{\arraystretch}{1.2}
		\begin{tabular}{lccc}
			Parameter						&	Value [dB]	& Design goal [dB] 	& Margin [dB]	\\
			\hline
			$\mathit{RL}_\mathrm{in}$		&	15.22		& $\geq 15$ 		& 0.22			\\
			$\mathit{RL}_\mathrm{out}$		&	$\infty$	& $\geq 15$  		& $\infty$		\\
			$G_\mathrm{A}$					&	13.35		& $\geq 13$  		& 0.35			\\				
			$G_\mathrm{T}$					&	13.35		& $\geq 13$  		& 0.35			\\
			$G_\mathrm{P}$					&	13.49		& $\geq 13$  		& 0.49			\\
			$F$								&	0.737		& $\leq 0.8$ 		& 0.062				
		\end{tabular}
	\end{center}
	\vspace{-1\oneLine}
\end{table}

\newpage
\vspace*{0pt}
\newpage
\section{Design}
\label{s:design}

After obtaining a preliminary starting point for the design it was time to pursue 
the actual design with ADS simulations. In ADS, the final balancing resistor, 
biasing networks and matching networks were to be designed and optimized. For 
the simulations, the used PCB material was assigned as RT-Duroid 5870 (Cu 1.0 oz/ft$^2$) 
with $\epsilon_\mathrm{r} = 2.33 \pm 0.02$, $h = 0.787 \pm 0.003$~mm, $t = 35$~um, and $\tan \delta = 0.0012$. 
The dimensions of the circuit board were $70 \times 85 \; \mathrm{mm}^2$. Fig. \ref{f:lo} 
presents the final amplifier layout. The layout consists of five individual 
sections, namely, the transistor section (a), input and output biasing networks 
(a,b) and input and output matching networks (d,e). 

\begin{figure}[!h]
	\centering
	\includegraphics[scale=1]{img/layout.png}
	\setlength{\unitlength}{1mm}
	\begin{picture}(1,1)
		\put(-32, 27.5){\small{(a)}}
		\put(-60.5, 58){\small{(b)}}
		\put(-26.5, 58){\small{(c)}}
		\put(-76, 17){\small{(d)}}
		\put(-9.4, 18){\small{(e)}}
		\put(-83, 26){\small{IN}}
		\put(-2.5, 23.5){\small{OUT}}
	\end{picture}
	\caption{The final amplifier layout comprising (a) transistor section, (b) input biasing, 
		(c) output biasing, (d) input matching and (e) output matching.}
	\label{f:lo}
\end{figure}


The transistor section includes the transistor followed by the stabilizing resistor. 
The two gaps at the input and output are for the DC blocking capacitors (10 pF, MURATA GQM series). 
The capacitors prevent the DC power from flowing to the input and output of the amplifier. 
The lines pointing vertically from the transistor are for the source. The three blue dots
on the source lines are the via-grounding holes. Using multiple via-holes lowers the inductance 
to the ground. An initial value for the balancing resistor was obtained from the preliminary 
design. The validity of the preliminary design was quickly checked with a S-parameter model 
of the transistor. The S-parameter model contains the S-parameters of a certainly biased 
transistor and thus does not require biasing networks. As the preliminary 100 $\Omega$ 
stabilizing resistor seemed valid it was time to replace the transistor with its non-linear 
model and continue with the design of the biasing networks. 

The biasing networks were designed separately and their operation was verified before attaching 
them to the actual amplifier model. The main purpose of a biasing networks is to provide DC 
voltage to empower the transistor. The DC power should not couple to the input and output of 
the amplifier nor should the RF signal flow to the DC source. Coupling can be avoided by, for 
instance, using DC blocks, radial stubs, signal absorbing resistors and shorting capacitors. 
A radial stub, placed at a distance of quarter wavelength from the original signal path, creates 
a virtual open circuit for the RF signal at a certain frequency. Thus the RF signal does not 
propagate to the DC source. The absorbing resistor and shorting capacitors are placed to account 
for the possible non-desired RF signals. The  resistor after the radial stub absorbs the possible 
reflections whereas the capacitors short the RF signals directly to the ground. 15 $\Omega$ 
absorbing resistors were found to be adequate. Three capacitors (1 pF, 10 pF and 100 pF, MURATA 
GQM series) were used for the shorting. Due to the non-ideality of the capacitors they provide 
an enhanced shorting to the ground. The grounding was made using four via-holes. The correct 
biasing gate to source voltage, $V_\mathrm{GS}$ was derived with an DC analysis in which the 
$V_\mathrm{GS}$ was swept to result in the required biasing current $I_\mathrm{DS}$. The 
result of the DC analysis is presented in Fig. \ref{f:bias}. Thus the correct biasing 
was achieved with $V_\mathrm{GS} = -0.39$~V. The voltage drop due to the absorbing 
resistor should be accounted by increasing the source DC voltage by the equivalent amount. In 
the input biasing network the voltage drop is insignificant due to the low current whereas at 
the output it is significant, namely, $15 \; \Omega \cdot 30 \; \mathrm{mA} = 0.45$~V.

\begin{figure}[!h]
	\centering
	\begin{tikzpicture}
		\begin{axis}[
		/pgf/number format/.cd,
		1000 sep={},
		ylabel=$I_\mathrm{DS} \, \mathrm{[mA]}$,
		xlabel=$V_\mathrm{DS} \, \mathrm{[V]}$,
		width=3.5in,
		height=1.86in,
		ymin=0,
		ymax=45,
		xmax=5,
		xmin=0,
		ytick={0,5,...,45},
		grid=both,
		%legend cell align=left,
		%legend style={
		%at={(0.01,0.02)},
		%anchor=south west}
		legend pos=south east
		]
		\addplot+[mark=none, color=red, solid] table[x=VDS,y=IDS_VGS_m44] {data/Simulations/Bias_point_all.txt};
		\addlegendentry{$V_\mathrm{GS} = -4.4$~V}
		\addplot+[mark=none, color=blue, solid] table[x=VDS,y=IDS_VGS_m39] {data/Simulations/Bias_point_all.txt};
		\addlegendentry{$V_\mathrm{GS} = -3.9$~V}
		\addplot+[mark=none, color=green, solid] table[x=VDS,y=IDS_VGS_m34] {data/Simulations/Bias_point_all.txt};
		\addlegendentry{$V_\mathrm{GS} = -3.4$~V}
		\addplot+[sharp plot, mark=none, color=black, solid] coordinates
		{(2,-1) (2,50)};
		\addplot+[sharp plot, mark=none, color=black, solid] coordinates
		{(-1,30) (6,30)};
		\end{axis}
	\end{tikzpicture}
	\caption{Bias point DC analysis.}
	\label{f:bias}
\end{figure}

After designing the biasing networks they were attached to the amplifier. 
Correctly designed biasing networks should provide similar simulation results 
as the S-parameter model of the transistor.

The input and output matching was done using a single open stub. The lengths 
of the transmission lines were tuned carefully using the Smith chart to meet 
the design criteria on the amplifier center frequency. At this point, the 
amplifier did not meet the set gain criteria and thus the value of the balancing 
resistor had to be decreased. A suitable compromise was found with a balancing 
resistor value of 50~$\Omega$. 

Fig. \ref{f:sim} presents the final simulation results for the input 
and output return losses $RL$, stability factor $K$, gain $G$ and noise figure 
$F$. The simulation results depict the amplifier to satisfy all of the set design 
specifications. The final simulation results at the design frequency at 2.5~GHz are summarized in Table \ref{t:sim results}.

\begin{figure}[h]
	\centering
	%\begin{center}
	\begin{tabular}{cc}
	\begin{tikzpicture}[baseline,trim axis left]
		\begin{axis}[
		/pgf/number format/.cd,
		1000 sep={},
		ylabel=$S_{ii} \, \mathrm{[dB]}$,
		xlabel=Frequency f $\mathrm{[GHz]}$,
		width=3in,
		height=1.86in,
		ymin=-35,
		ymax=5,
		xmax=3.5,
		xmin=1.5,
		ytick={-35,-30,...,5},
		grid=both,
		%legend cell align=left,
		%legend style={
		%at={(0.01,0.02)},
		%anchor=south west}
		legend pos=south west
		]
		\addplot+[mark=none, color=red, solid] table[x=freq,y=S11] {data/Simulations/Simulation_results_all.txt};
		\addlegendentry{$S_{11}$}
		\addplot+[mark=none, color=blue, solid] table[x=freq,y=S22] {data/Simulations/Simulation_results_all.txt};
		\addlegendentry{$S_{22}$}

		\end{axis}
	\end{tikzpicture}

	&

	\hspace{25pt}
	\begin{tikzpicture}[baseline,trim axis left]
		\begin{axis}[
		/pgf/number format/.cd,
		1000 sep={},
		ylabel=Stability factor $\mu$,
		xlabel=Frequency f $\mathrm{[GHz]}$,
		width=3in,
		height=1.86in,
		ymin=0,
		ymax=2.5,
		xmax=3.5,
		xmin=1.5,
		ytick={0,0.5,...,2.5},
		grid=both,
		%legend cell align=left,
		%legend style={
		%at={(0.01,0.02)},
		%anchor=south west}
		%legend pos=south east
		]
		\addplot+[mark=none, color=red, solid] table[x=freq,y=Stability] {data/Simulations/Simulation_results_all.txt};

		\end{axis}
	\end{tikzpicture}
	\\
	%
	\begin{tikzpicture}[baseline,trim axis left]
		\begin{axis}[
		/pgf/number format/.cd,
		1000 sep={},
		ylabel=Gain $G$ $\mathrm{[dB]}$,
		xlabel=Frequency f $\mathrm{[GHz]}$,
		width=3in,
		height=1.86in,
		ymin=0,
		ymax=15,
		xmax=3.5,
		xmin=1.5,
		ytick={0,3,...,15},
		grid=both,
		%legend cell align=left,
		%legend style={
		%at={(0.01,0.02)},
		%anchor=south west}
		%legend pos=south east
		]
		\addplot+[mark=none, color=red, solid] table[x=freq,y=Gain] {data/Simulations/Simulation_results_all.txt};

		\end{axis}
	\end{tikzpicture}

	&

	\hspace{25pt}
	\begin{tikzpicture}[baseline,trim axis left]
		\begin{axis}[
		/pgf/number format/.cd,
		1000 sep={},
		ylabel=Noise figure $F$ $\mathrm{[dB]}$,
		xlabel=Frequency f $\mathrm{[GHz]}$,
		width=3in,
		height=1.86in,
		ymin=0,
		ymax=6,
		xmax=3.5,
		xmin=1.5,
		ytick={0,1,...,6},
		grid=both,
		%legend cell align=left,
		%legend style={
		%at={(0.01,0.02)},
		%anchor=south west}
		%legend pos=south east
		]
		\addplot+[mark=none, color=red, solid] table[x=freq,y=Noise] {data/Simulations/Simulation_results_all.txt};

		\end{axis}
	\end{tikzpicture}
	\\
	\end{tabular}
	\caption{Simulation results for input and output return losses, stability factor, gain and noise.}
	\label{f:sim}
\end{figure}

%Table summarizing results at 2.5 GHz

%$\textit{RL}_\mathrm{in} = 22.5 \;\dB$, $\textit{RL}_\mathrm{out} = 29.5 \;\dB$, 
%$K = 1.50$, $G = 13.86 \;\dB$ and $F = 0.53 \;\dB$.

\begin{table}[h]
%\renewcommand{\arraystretch}{1.3}
\caption{Simulation results at 2.5 GHz}
\label{t:sim results}
\centering
%more advanced table example
%\begin{tabular}{|l|c|c|c|c|}
%\hline
%\multirow{2}{*}{\parbox[t]{0.6in}{\bfseries Theoretical\\Read Range} } & \multicolumn{2}{c|}{ \bfseries EPCglobal Standard} & \multicolumn{2}{c|}{ \bfseries Built System} \\ \cline{2-5}
%& \bfseries 16 mm book & \bfseries 2 books & \bfseries 16 mm book & \bfseries 2 books \\ \hline \hline
%\bfseries Forward [m] & \bfseries 7.9 & \bfseries 8.5 & 7.9 & 8.5 \\ \cline{2-3} \hline
%\bfseries Reverse [m] & 10.9 & 9.8 & \bfseries 4.3 & \bfseries 3.9 \\ \cline{2-3} \hline
%\end{tabular}
\begin{tabular}{lc}
Parameter & Value \\ \hline
$RL_\mathrm{in} \; \mathrm{[dB]}$ & 22.5 \\ 
$RL_\mathrm{out} \; \mathrm{[dB]}$ & 29.5 \\ 
$K$ & 1.5 \\ 
$G \; \mathrm{[dB]}$ & 13.86 \\ 
$F \; \mathrm{[dB]}$ & 0.53 \\ 
\end{tabular}
\end{table} 


Fig. \ref{f:hb_gain} and \ref{f:hb_toi} illustrate the 1-dB gain compression and third order intercept 
(TOI) point for the amplifier. The second and third order intermodulation terms ($\textit{IM}_2$ and $\textit{IM}_3$) 
are also plotted in addition to the desired signal frequency. The 1-dB gain compression and TOI occur 
at input power levels of $-1.3$~dB and 6.7~dB, respectively. 

\begin{figure}[!h]
	\centering
	%\begin{tabular}{cc}
	\begin{tikzpicture}%[baseline, trim axis left]
		\begin{axis}[
		/pgf/number format/.cd,
		1000 sep={},
		ylabel=Gain G $\mathrm{[dB]}$,
		xlabel=$P_\mathrm{in} \; \mathrm{[dBm]}$,
		width=5in,
		height=3.6in,
		ymin=0,
		ymax=15,
		xmax=15,
		xmin=-30,
		ytick={0,3,...,15},
		grid=both,
		%legend cell align=left,
		%legend style={
		%at={(0.01,0.02)},
		%anchor=south west}
		%legend pos=south east
		]
		\addplot+[mark=none, color=red, solid] table[x=P_in,y=Gain] {data/Simulations/Simulation_HB_results_all.txt};

		\addplot [black, mark=*, mark options={draw=black}]
		coordinates {
		(-1.3,12.8589) 
		};

		\node at (axis cs:-1.3,12.8589) [anchor=south west] {1-dB compression};

		\end{axis}
	\end{tikzpicture}
	\caption{Simulation results for gain compression and the 1-dB compression point.}
	\label{f:hb_gain}
\end{figure}	
	%&
\begin{figure}[!h]	
	\centering
	\begin{tikzpicture}%[baseline]
		\begin{axis}[
		/pgf/number format/.cd,
		1000 sep={},
		ylabel=$V_\mathrm{out} \; \mathrm{[dBm]}$,
		xlabel=$P_\mathrm{in} \; \mathrm{[dBm]}$,
		width=5in,
		height=3.6in,
		ymin=-85,
		ymax=35,
		xmax=15,
		xmin=-30,
		ytick={-85,-65,...,35},
		grid=both,
		legend cell align=left,
		%legend style={
		%at={(0,1.02)},
		%anchor=south west}
		legend pos=south east
		]
		\addplot+[mark=none, color=red, solid] table[x=P_in,y=V_out] {data/Simulations/Simulation_HB_results_all.txt};
		\addlegendentry{Fundamental}
		\addplot+[mark=none, color=green, dashed] table[x=P_in,y=V_out_2H] {data/Simulations/Simulation_HB_results_all.txt};
		\addlegendentry{$\mathit{IM}_2$}
		\addplot+[mark=none, color=blue, densely dashed] table[x=P_in,y=V_out_3H] {data/Simulations/Simulation_HB_results_all.txt};
		\addlegendentry{$\mathit{IM}_3$}

		\addplot+[mark=none, color=red, densely dotted] table[
		x=P_in,
		y={create col/linear regression={y=V_out, 
		%variance list is used to weight the selection of points. Dummy solution. Not good.
		variance list={1,1,1,1,1,1,1,1,1,1,
		1,1,1,1,1,1,1,1,1,1,
		1,1,1,1,1,1000,
		1000,1000,1000,1000,1000,1000,1000,1000,1000,1000,
		1000,1000,1000,1000,1000,1000,1000,1000,1000,1000}
		}}]
		{data/Simulations/Simulation_HB_results_all.txt};

		\addplot+[mark=none, color=blue, densely dotted] table[
		x=P_in,
		y={create col/linear regression={y=V_out_3H,
		%variance list is used to weight the selection of points. Dummy solution. Not good. 
		variance list={1000,1000,1000,1000,1000,1000,1000,1000,1000,1000,
		1000,1000,1000,1000,1000,1000,1000,1000,1000,1000,
		1000,1000,1000,1000,1000,1,1,1,1,1,1,1000,1000,1000,1000,1000,
		1000,1000,1000,1000,1000,1000,1000,1000,1000,1000}
		}}]
		{data/Simulations/Simulation_HB_results_all.txt};

		\addplot [black, mark=*, mark options={draw=black}]
		coordinates {
		(6.7,20.5) 
		};

		\node at (axis cs:6.7,20.5) [anchor=south east] {TOI};

		\end{axis}
	\end{tikzpicture}
	%\\
	%\end{tabular}

	\caption{Simulation results for intermodulation terms and TOI.}
	\label{f:hb_toi}
\end{figure}

Fig. \ref{f:proto} presents the final manufactured amplifier with the surface mounted components and connectors soldered in place. 

\begin{figure}[!h]
	\begin{center}
	%\centering
	\includegraphics[width=4in]{img/proto}
	%\setlength{\unitlength}{1mm}
	\caption{The final manufactured proto amplifier.}
	\label{f:proto}
	\end{center}
\end{figure}

\newpage
\section{Measurements}

The measurements began by determining the S-parameters of the amplifier. A VNA was used for this purpose which was first calibrated with a SOLT calibration kit. The amplifier was biased using two power sources. First, the simulated voltage gate of approximately -0.39 V was applied to the gate and then the drain voltage was increased slowly. It was soon realized that the previously simulated and calculated biasing point could not be reached. To reach the required $I_\mathrm{DS}=30 \; \mathrm{mA}$ the $V_\mathrm{DS}$ had to be increased to 4.2 V. This difference could partly result from the inaccurate adjustability of the power sources. After biasing the amplifier the s-parameters were measured. The measured S-parameters are presented in Fig. \ref{m:S-parameters}. The S-parameter results depict a shift of the amplifier center frequency to lower frequencies.  

\begin{figure}[!h]
	\begin{center}
	%\centering
	\includegraphics[width=5in]{data/Measurements/sparam.pdf}
	%\setlength{\unitlength}{1mm}
	\caption{The measured S-paramaters $S_{11}$, $S_{22}$ and $S_{21}$.}
	\label{m:S-parameters}
	\end{center}
\end{figure}

The gain performance is mostly set by the biasing point and is still quite acceptable at the original design frequency. On the contrary, the input and output matching are clearly shifted to lower frequencies, namely to 2 GHz. The input and output reflection coefficients can be inspected on the Smith chart to get a hint of this frequency shifting. The input and output reflection coefficients are plotted in Fig. \ref{m:Smith_chart} at frequencies 1.5 GHz to 3.5 GHz. The information on the Smith chart does not propose any reasonable method for improving the input matching. In contrast, the output matching seems to be rotated by a possibly unaccounted additional parallel capacitance. Series inductance would rotate the matching similarly. The output matching could possibly be improved by disabling some of the via-grounds which would increase the parallel inductance thus rotating the matching point towards the center of the Smith chart. The measurements presented later on are conducted at 2 GHz as the built amplifier's performance would be really poor on the desired 2.5 GHz.

%SMITHIN KARTTA-----------------------------------------------------------

\begin{figure}[h]
\centering
\begin{tikzpicture}
\begin{smithchart}[
%legend cell align=left,
%legend style={
%at={(0.3,0.8)},
%anchor=south east}
legend pos=north west
]
%\addplot coordinates {(0.5,0.2) (1,0.8) (2,2)};
\addplot[red, solid,is smithchart cs]
file {Data/Measurements/Smith_S11.txt};
\addlegendentry{$S_{11}$}
\addplot[blue, solid,is smithchart cs]
file {Data/Measurements/Smith_S22.txt};
\addlegendentry{$S_{22}$}
%\addplot[blue, densely dotted,is smithchart cs]
%file {Data/SmithData_er185.txt};
%\addlegendentry{$\epsilon_\mathrm{r}=1.85$}
%\addplot[brown, densely dashed,is smithchart cs]
%file {Data/SmithData_er285.txt};
%\addlegendentry{$\epsilon_\mathrm{r}=2.85$}
\end{smithchart}
\end{tikzpicture}
\caption{Input and output reflection coefficients at frequencies 1.5 GHz to 3.5 GHz.
%Effect of books relative permittivity to the resonance frequency and matching. Book thickness or the amount of paper and adjustment parameter \textit{dw} rotates the double resonance in a similar manner.
\label{m:Smith_chart}}
\end{figure}

Fig. \ref{m:gain} presents the gain saturation and the 1-dB compression point of the amplifier at 2 GHz. The unsaturated gain is 14.1 dB and the 1-dB compression occurs at an input power of approximately -2.9 dB. 

\begin{figure}[!h]
	\begin{center}
	%\centering
	\includegraphics[width=5in]{data/Measurements/icp.pdf}
	%\setlength{\unitlength}{1mm}
	\caption{The measured gain saturation and 1-dB compression point.}
	\label{m:gain}
	\end{center}
\end{figure}

Intermodulation of the amplifier was inspected by applying two closely separated signal frequencies to the input of the amplifier and setting their power intensity to the same level. The signal frequencies were chosen as 1.999 GHz and 2.001 GHz. Thus the inspected third order intermodulation term would occur at 2.003 GHz. The intermodulation term was measured as the original signal frequency power levels were increased. The resulting plot is presented in Fig.~\ref{m:toi}. The TOI is approximately at an input power level of 3 dB.

\begin{figure}[!h]
	\begin{center}
	%\centering
	\includegraphics[width=5in]{data/Measurements/toi.pdf}
	%\setlength{\unitlength}{1mm}
	\caption{Result of the intermodulation measurement.}
	\label{m:toi}
	\end{center}
\end{figure}

An important occurrence was observed when conducting the intermodulation measurement originally. The spectrum analyzer displayed 10 MHz harmonics around the original signal frequencies. This indicated that the amplifier input either had some disturbing signal present or that the transistor was damaged. The situation was fixed by replacing the transistor and adding an additional 1 uF capacitor to the output biasing ground. Most likely the transistor was broken. Transistors are most likely broken by the excessive heat during soldering and de-soldering. Applying the biasing drain voltage before the gate voltage or exceeding the maximum allowed current of the transistor may also damage it (even burn it, as was noticed).

Noise figure of the amplifier was measured by using the Y-factor method. In the Y-factor method a noise source is connected to the input of the amplifier. The noise source is used as a hot and cold source by applying a biasing voltage and the output power from the amplifier is measured. In this case the output noise power is too low and requires an additional pre-amplifier. Thus the Y-factor method has to be applied twice by first measuring the hot and cold source only with the pre-amplifier applied. When the DUT (built amplifier) is in conjunction with the pre-amplifier the DUT noise figure can be calculated from the cascaded system.



\newpage
\section{Results}

Text

\newpage
\section{Conclusions}

Text

\newpage
\section{Feedback}

Text


\newpage
\begin{thebibliography}{9}%\itemsep 7pt\parskip -5pt 
	
\bibitem{bahl} I.\ J.\ Bahl, 
	\textit{Fundamentals of RF and Microwave Transistor Amplifiers},
	J.\ Wiley \& Sons, 2009.

\bibitem{gonz} G.\ Gonzalez, 
	\textit{Microwave Transistor Amplifiers -- Analysis and Design},
	Prentice Hall, 2nd Ed., 1997.
	
\bibitem{pozar} D.\ M.\ Pozar, 
	\textit{Microwave Engineering}, 
	J.\ Wiley \& Sons, 4th Ed., 2012.
	
\bibitem{lsh} C.\ Icheln (edited), 
	\textit{Lecture supplement handout},
	S-26.3120 Laboratory course in Radio Engineering course material.

\end{thebibliography}

\end{document}
